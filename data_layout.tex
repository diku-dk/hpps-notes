\lstset{language=C}

\chapter{Data Layout}
\label{chap:datalayout}

One of the things that makes C a difficult programming language is
that it does not provide many built-in data types, and provides poor
support for making it convenient to work with new data types.  In
particular, C has notoriously poor support for multi-dimensional
arrays.  Given that multi-dimensional arrays are perhaps the single
most important data structure for scientific computing, this is not
good.  In this chapter we will look at how we \textit{encode}
mathematical objects such as matrices (two-dimensional arrays) with
the tools that C makes available to us.  One key point is that there
are often multiple ways to represent the same object, with different
pros and cons, depending on the situation.

\section{Arrays in C}

At the surface level, C does support arrays.  We can declare a
$n\times{}m$ array as
\begin{lstlisting}
double A[n][m];
\end{lstlisting}
and then use the fairly straightforward \lstinline{A[i][j]} syntax to
read a given element.  However, C's arrays are a second-class language
construct in many ways:

\begin{itemize}
\item They decay to pointers in many situations.
\item They cannot be passed to a function without ``losing'' their
  size.
\item They cannot be returned from a function at all.
\end{itemize}

In practice, we tend to only use language-level arrays in very simple
cases, where the sizes are statically known, and they are not passed
to or from functions.  For general-purpose usage, we instead build our
own representation of multi-dimensional arrays, using C's support for
\textit{pointers} and \textit{dynamic allocation}.  Since actual
machine memory is essentially a single-dimensional array, working with
multi-dimensional arrays in C really just requires us to answer one
central question:
\begin{center}
  \textbf{How do we map a multi-dimensional index to a
    single-dimensional index?}
\end{center}
Or to put it another way, representing a $d$-dimensional array in C
requires us to define a bijective\footnote{A bijective function is a
  function between two sets that maps each element of each set to a
  distinct element of the other set.} \textit{index function}
\begin{equation}
  I : \mathbb{N}^{d} \rightarrow \mathbb{N}
\end{equation}

The index function maps from our (mathematical, conceptual)
multi-dimensional space to the one-dimensional memory space offered by
an actual computer.  This is sometimes also called \textit{unranking},
although this is strictly speaking a more general term from
combinatorics.

As an example, suppose we wish to represent the following $3\times{}4$ matrix in memory:

\begin{equation}
  \begin{pmatrix}
    11 & 12 & 13 & 14 \\
    21 & 22 & 23 & 24 \\
    31 & 32 & 33 & 34
  \end{pmatrix}
\label{eqn:matA}
\end{equation}

We can do this in any baroque way we wish, but the two most common
representations are:

\begin{description}
\item[\textbf{Row-major order},] where elements of each \textit{row} are
  contiguous in memory:

  \begin{center}
  \begin{tabular}{|c|c|c|c|c|c|c|c|c|c|c|c|}
    \hline
    11&12&13&14&21&22&23&24&31&32&33&34\\
    \hline
  \end{tabular}
  \end{center}

  with index function
  \[
    (i,j) \mapsto i\times 4 + j
  \]
\item[\textbf{Column-major order},] where elements of each \textit{column}
  are contiguous in memory:

  \begin{center}
  \begin{tabular}{|c|c|c|c|c|c|c|c|c|c|c|c|}
    \hline
    11&21&31&12&22&32&13&23&33&14&24&34\\
    \hline
  \end{tabular}
  \end{center}

  with index function
  \[
    (i,j) \mapsto j\times 3 + i
  \]
\end{description}

The index functions are generalised on \cref{fig:indexfunctions}.
Note that the two representations contain the exact same values, so
they encode the same mathematical object, but in different ways.  The
intuition for the row-major index function is that we first skip $i$
rows ahead to get to the row of interest, then move $j$ columns into
the row.

Row-major order is used by default in most programming languages and
libraries, but not universally so---the scientific language Fortran is
famously column-major.  The NumPy library for Python uses row-major by
default (called \texttt{C} in Numpy), but one can explicitly ask for
arrays in column-major order (called \texttt{F}), which is sometimes
needed when exchanging data with systems that expect a different
representation.

\begin{figure*}
  \centering

  \begin{subfigure}[b]{0.45\textwidth}
    \begin{equation}
      (i,j) \mapsto i\times m + j \label{eqn:idx2row}
    \end{equation}
    \caption{Row-major indexing.}
    \label{fig:rowmajor}
  \end{subfigure}
  \hfill
  \begin{subfigure}[b]{0.45\textwidth}
    \begin{equation}
      (i,j) \mapsto j\times n + i \label{eqn:idx2col}
    \end{equation}
    \caption{Column-major indexing.}
    \label{fig:colmajor}
  \end{subfigure}

  \caption{Index functions for $n\times{}m$ arrays represented in
    row-major and column-major order.  For an example of why computer
    scientists tend to prefer 0-indexing, try rewriting the above to
    work with 1-index arrays instead.}
  \label{fig:indexfunctions}
\end{figure*}

\subsection{Implementation in C}

Let's look at how to implement this in C.  Let's say we wish to
represent the matrix from \cref{eqn:matA} in row-major order.  Then we
would write the following (assuming \texttt{n=3}, \texttt{m=4}):

\begin{lstlisting}
int *A = malloc(n*m*sizeof(int));
A[0] = 11;
A[1] = 12;
...
A[11] = 34;
\end{lstlisting}

Note that even though we \textit{conceptually} wish to represent a
two-dimensional array, the actual C type is technically a
single-dimensional array with 12 elements.  If we when wish to index
at position $(i,j)$ we then use the expression \lstinline{A[i*4+j]}.

Similarly, if we wished to use column-major order, we would program as
follows:

\begin{lstlisting}
int *A = malloc(n*m*sizeof(int));
A[0] = 11;
A[1] = 21;
...
A[11] = 34;
\end{lstlisting}

To C there is no difference---and there is no indication in the types
what we intended.  This makes it very easy to make mistakes.

Note also how it is on \textit{us} to keep track of the sizes of the
array---C is no help.  Don't make the mistake of thinking that
\lstinline{sizeof(A)} will tell you how big this array is---while C
will produce a number for you, it will indicate the size of a
\textit{pointer} (probably 8 on your machine).

Some C programmers like defining functions to help them generate the
flat indexes when indexing arrays:

\begin{lstlisting}
int idx2_rowmajor(int n, int m, int i, int j) {
  return i * m + j;
}

int idx2_colmajor(int n, int m, int i, int j) {
  return j * n + i;
}
\end{lstlisting}

Note how row-major indexing does not use the \texttt{n} parameter, and
column-major indexing does not use \texttt{m}.

However, these functions do not on their own fully prevent us from
making mistakes.  Consider indexing the \texttt{A} array from before
with the expression

\begin{center}
\lstinline{A[idx2_rowmajor(n, m, 2, 5)]}.
\end{center}

Here we are trying to access index $(2,5)$ in a $3\times{}4$
array---which is conceptually an out-of-bounds access.  However, by
the index function, this translates to the flat index
$2\times{}3+5=11$, which is in-bounds for the 12-element array we use
for our representation in C.  This means that handy tools like
\texttt{valgrind} will not even be able to detect our mistake---from
C's point of view, we're doing nothing wrong!  Things like this make
scientific computing in C a risky endeavour.  We can protect ourselves
by using helper functions like those above, and augment them with
\texttt{assert} statements that check for problems:

\begin{lstlisting}
int idx2_rowmajor(int n, int m, int i, int j) {
  assert(i >= 0 && i < n);
  assert(j >= 0 && j < m);
  return i * m + j;
}
\end{lstlisting}

We can still make mistakes, but at least now they will be noisy,
rather than silently reading (or corrupting!) unintended data.

\subsection{Size passing}
\label{sec:sizepassing}

With the previously discussed representation, a multidimensional array
(e.g. a matrix) is just a pointer to the data, along with metadata
about its size.  The C language does not help us keep this metadata in
sync with reality.  When passing one of these arrays to a function, we
must manually pass along the sizes, and we must get them right without
much help from the compiler.  For example, consider a function that
sums each row of a (row-major) $n\times{}m$ array, saving the results
to an $n$-element output array:

\lstinputlisting[
caption={Summing the rows of a matrix.},
label={lst:sumrows.c},
language=C,
frame=single]
{src/sumrows.c}

C gives us the raw building blocks of efficient computation, but we
must put together the pieces ourselves.  We protect ourselves by
carefully documenting the data layout expected of the various
functions.  For the \texttt{sumrows} function above, we would document
that \texttt{matrix} is expected to be a row-major array of size
$n\times{}m$.

\subsection{Slicing}
\label{sec:slicing}

In high-level languages like Python, we can use notation such as
\texttt{A[i:j]} to extract a \textit{slice} (a contiguous subsequence)
of an array, in this case of the elements from index \texttt{i} to
\texttt{j}.  No such syntactical niceties are available in C, but by
using our knowledge of how arrays are physically laid out in memory,
we can obtain a similar effect in many cases.

Suppose \texttt{V} is a vector of \texttt{n} elements, and we wish to
obtain a slice of the elements from index \texttt{i} to \texttt{j}
(the latter exclusive).  In Python, we would merely write
\texttt{V[i:j]}.  In C, we compute the size of the slice as
\begin{lstlisting}
int m = j - i;
\end{lstlisting}
and then compute a pointer to the start of the slice:
\begin{lstlisting}
double *slice = &V[i];
\end{lstlisting}
Now we can treat \texttt{slice} as an \texttt{m}-element array that
just happens to use the same underlying storage as \texttt{V}.  This
means that we must be careful not to deallocate \texttt{V} while
\texttt{slice} is still in use.

Similarly, if \texttt{A} represents a matrix of size \texttt{n} by
\texttt{m} in row-major order, then we can produce a vector
representing the \texttt{i}th row as follows:
\begin{lstlisting}
double *row = &A[i*m];
\end{lstlisting}
The restriction is that such slicing can only produce arrays whose
elements are \emph{contiguous} in memory.  For example, we cannot
easily extract a column of a row-major array, because the elements of
a column are not contiguous in memory.  If we wish to extract a
column, then we have to allocate space and copy element-by-element, in
a loop\footnote{There are more sophisticated array representations
  that use \textit{strides} to allow array elements that are not
  contiguous in memory---NumPy uses these, but they are outside the
  scope of our course.}.

\subsection{Even higher dimensions}

The examples so far have focused on the two-dimensional case.
However, the notion of row-major and column-major order generalises
just fine to higher dimensions.  The key distinction is that in a row-major
array, the \textit{last} dimension is contiguous, while for a
column-major array, the \textit{first} dimension is contiguous.  For a
row-major array of shape $n_{0} \times{} \cdots \times{} n_{d-1}$, the
index function where $p$ is a $d$-dimensional index point is

\begin{equation}
  p \mapsto \sum_{0 \leq i < d} p_{i} \prod_{i<j<d} n_{j}
  \label{eqn:idxDrow}
\end{equation}

where $p_{i}$ gets the $i$th coordinate of $p$, and the product of an
empty series is $1$.

Similarly, for a column-major array, the index function is

\begin{equation}
  p \mapsto \sum_{0 \leq i < d} p_{i} \prod_{0\leq j<i} n_{j}
\end{equation}

We can also have more complex cases, such as a three-dimensional array
where the two-dimensional ``rows'' are stored consecutively, but are
individually column-major.  Such constructions can be useful, but are
beyond the scope of this course (and are a nightmare to implement).

%%% Local Variables:
%%% mode: latex
%%% TeX-master: "hpps-notes"
%%% End:
